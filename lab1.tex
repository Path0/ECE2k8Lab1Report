%Template generated by Ben Manning
%Purdue University
%btmannin@purdue.edu
%Last modified: 7/7/2021


\documentclass[notitlepage, 12pt]{report}  %The document class will setup a lot of basic formatting.  Report class will start with justifying paragraphs setup your different types of sections.

%Different packages allow you to add more functions that will make your time in LaTeX easier.
\usepackage{amsmath}
\usepackage{graphicx} % pictures
\usepackage{caption}
\usepackage{url}
\usepackage{circuitikz} % circuit drawer

%\usepackage{biblatex} %Imports biblatex package
\usepackage[style=numeric]{biblatex}
\addbibresource{bib.bib}

\usepackage[top=2cm, bottom=2cm, left=2cm, right=2cm]{geometry}

\title{Experiment 1 Report}

\begin{document}
%Everything needs to begin and end.  


\begin{center}
\large \textbf{Experiment 1 Report} \\ %\large and \small can help make text sizes vary throughout your document.
%\textbf will bold the text that is in the curly brackets
\small 
Andrew Lykken\\
Anna Kishnani\\
19 January 2023\\
Section 004 (Abraham Yakisan)\\
%\rule{500pt}{.1pt} 

\end{center}

% space between tite and abstract
\vspace{4in}


\begin{abstract}
abstract here 
\end{abstract}

\newpage

\section*{Task 1} %Each task has a section including (but not limited to) Objective, 
% Procedure, Results / Calculations, Conclusions


\subsection*{Objective}
This section will demonstrate basic measurement techniques and functions learned in ECE 20007, using 
the lab instruments provided, including an oscilloscope, waveform generator, digital multimeter (DMM), 
and variable DC power supply. 

\subsection*{Procedure}
\indent\indent For testing purposes, the function:
\begin{equation}
    2 \sin(2\pi 750t)
\end{equation}

was generated with the waveform generator.\\

The waveform generator was connected to one channel of an oscilloscope, and adjusted so that the signal 
was stable and reasonably sized within the window.\\


\noindent On the oscilloscope, 

\begin{itemize} % bullet points

    \item{The window was resized horizontally such that more periods of the wave were visible.}

    \item{The trigger level was modified between -4V and 4V.}

    \item{The horizontal and vertical offsets of the signal were modified.}
    
\end{itemize}

After recording findings, the DC power supply was set to 3.3V. The AC and DC RMS values were measured through 
the DMM.\\


\subsection*{Results / Calculations}

\indent\indent Each control serves a different function for manipulating the signal viewed on the oscilloscope. \\

The horizontal scale adjustment knob determines the scale of time on the oscilloscope's horizontal axis. 
This can vary widely, depending on what signal is being passed to the oscilloscope's input channel. 
An analogy for what this control does is that it ``squishes'' or ``stretches'' the view of the signal being measured. \\

The trigger level is a set voltage that the oscilloscope looks for when measuring signals to create stability in the
image shown. The level of the trigger determines at what voltage and time the signal lines up with. 

The offsets control the position of the signal in the viewing window. The vertical offset can move the signal up 
or down on the graph, where ground is shown with a marker on the vertical axis to the left of the signal. The
horizontal offset moves the signal in time as shown. The signal appears to move left or right, indicating a 
change in time measurements, being slightly ahead or behind where the original location was. \\




\subsection*{Conclusions}




\newpage

\printbibliography[title={\Large References}] %Prints out the bibliography sources that you have used in the document.

\end{document}