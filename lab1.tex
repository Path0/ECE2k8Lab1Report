%Template generated by Ben Manning
%Purdue University
%btmannin@purdue.edu
%Last modified: 7/7/2021


\documentclass[notitlepage, 12pt]{report}  %The document class will setup a lot of basic formatting.  Report class will start with justifying paragraphs setup your different types of sections.

%Different packages allow you to add more functions that will make your time in LaTeX easier.
\usepackage{amsmath}
\usepackage{graphicx} % pictures
\usepackage{caption}
\usepackage{url}
\usepackage{circuitikz} % circuit drawer

%\usepackage{biblatex} %Imports biblatex package
\usepackage[style=numeric]{biblatex}
\addbibresource{bib.bib}

\usepackage[top=2cm, bottom=2cm, left=2cm, right=2cm]{geometry}

\graphicspath{ {./images/} }

\title{Experiment 1 Report}

\begin{document}
%Everything needs to begin and end.  


\begin{center}
\large \textbf{Experiment 1 Report} \\ %\large and \small can help make text sizes vary throughout your document.
%\textbf will bold the text that is in the curly brackets
\small 
Andrew Lykken\\
Anna Kishnani\\
19 January 2023\\
Section 004 (Abraham Yakisan)\\
%\rule{500pt}{.1pt} 

\end{center}

% space between title and abstract
\vspace{5in}


\begin{abstract}

This experiment demonstrates the CD4007UB integrated circuit and its functionality. The CD4007UB is a dual complimentary pair 
plus inverter chip, containing 3 pMOSFET transistors and 3 nMOSFET transistors. This experiment tests the chip by determining 
the voltage transfer characteristic of the inverter in the chip to properly examine its properties and various voltage thresholds.
The experiment later uses the CD4007UB to create a digital ring oscillator to measure the propagation delay of the inverters. \\

The experiment found that the transfer characteristic appeared as expected, with voltage properties being within reason for
the chip design. When oriented as a ring oscillator, the performance remained good, with oscillation frequency and 
propagation delay being within the limits set by the manufacturer as stated on the product datasheet.



\end{abstract}

\newpage

% ----------------BEGIN TASKS--------------------- %

% ----- TASK 1 ----- %

\section*{Task 1} %Each task has a section including (but not limited to) Objective, 
% Procedure, Results / Calculations, Conclusions


\subsection*{Objective}
\indent\indent This section will demonstrate basic measurement techniques and functions learned in ECE 20007, using 
the lab instruments provided, including an oscilloscope, waveform generator, digital multimeter (DMM), 
and variable DC power supply. 

\subsection*{Procedure}
\indent\indent For testing purposes, the function:
\begin{equation}
    2 \sin(2\pi 750t)
\end{equation}

was generated with the waveform generator.\\

The waveform generator was connected to one channel of an oscilloscope, and adjusted so that the signal 
was stable and reasonably sized within the window.\\


\noindent On the oscilloscope, 

\begin{itemize} % bullet points

    \item{The window was resized horizontally such that more periods of the wave were visible.}

    \item{The trigger level was modified between -4V and 4V.}

    \item{The horizontal and vertical offsets of the signal were modified.}
    
\end{itemize}

After recording findings, the DC power supply was set to 3.3V. The AC and DC RMS values were measured through 
the DMM.\\


\subsection*{Results / Calculations}

\indent\indent Each control serves a different function for manipulating the signal viewed on the oscilloscope. \\

The horizontal scale adjustment knob determines the scale of time on the oscilloscope's horizontal axis. 
This can vary widely, depending on what signal is being passed to the oscilloscope's input channel. 
An analogy for what this control does is that it ``squishes'' or ``stretches'' the view of the signal being measured. \\

The trigger level is a set voltage that the oscilloscope looks for when measuring signals to create stability in the
image shown. The level of the trigger determines at what voltage and time the signal lines up with.\\

\newpage

Below is a screenshot containing the signal displayed in normal mode.

\begin{center}
    \includegraphics[scale=0.5]{scope1.png}
\end{center}


The offsets control the position of the signal in the viewing window. The vertical offset can move the signal up 
or down on the graph, where ground is shown with a marker on the vertical axis to the left of the signal. The
horizontal offset moves the signal in time as shown. The signal appears to move left or right, indicating a 
change in time measurements, being slightly ahead or behind where the original location was. \\

For using the DMM and DC power supply, the DC RMS and AC RMS values were measured to be:

\begin{equation}
    DC_{RMS} = 3.3005V
\end{equation}

\begin{equation}
    AC_{RMS} = 0.003V
\end{equation}

\vspace{7pt}

Knowing the nominal value for the $DC_{RMS}$ being 3.3V, the following error calculation can be achieved:

\begin{equation}
    \%_{err} = \frac{|V_{meas} - V_{ideal}|}{V_{ideal}} = \frac{|DC_{RMS,meas} - DC_{RMS,nominal}|}{DC_{RMS,nominal}} = 
    \frac{|3.305V - 3.3V|}{3.3V} = 0.15\%
\end{equation}


\subsection*{Conclusions}

\indent\indent This task explored the techniques and adjustments that can be made when measuring a circuit or signal,
and how such measurements can impact the readability and usefulness of the tools used. On the oscilloscope, 
signals can be manipulated in many ways to make them more readable and measurable to a user. The DMM has
multiple modes that can be used to measure different types of signals, differentiated by being AC or DC 
values.\\


% ----- TASK 2 ----- %

\section*{Task 2}

\subsection*{Objective}
\indent\indent This section will explore a CMOS inverter and its voltage transfer characteristic. 
\subsection*{Procedure}
\textbf{Steps 1 - 5}\\

The following circuit was built:


\begin{center}
    \includegraphics[scale=0.35]{singleInverter.png}
\end{center}

using the CD4007UB integrated circuit, and measuring $v_{in}$ at the function generator and $v_{out}$ at pin 12 and ground.\\
$V_{DD}$ was set to 5V DC, and $v_{in}$ was a 500 Hz 0V to 5V triangle wave.\\

\textbf{Step 6}\\
The oscilloscope display was captured showing $v_{in}$ and $v_{out}$ as signals. The oscilloscope display was then changed 
to be in XY mode, such that $v_{in}$ represented the horizontal axis, and $v_{out}$ represented the vertical axis.\\

\newpage

\subsection*{Results / Calculations}

\textbf{Step 6}\\

The following screenshot shows $v_{in}$ and $v_{out}$ displayed as 2 channels in normal time mode.\\

\begin{center}
    \includegraphics[scale=0.5]{scope2.png}
\end{center}

\newpage

\textbf{Step 7}\\

\vspace{4pt}
The below graph represents the transfer characteristic $v_{out}$ vs. $v_{in}$, with points labeled to represent
what mode the transistor is in, and which voltages correlate with those descriptions.\\

\begin{center}
    \includegraphics[scale=0.4]{scope3.png}
\end{center}

\vspace{7pt}

\newpage

\textbf{Step 8}\\

The below graph depicts the transfer characteristic ${v_{out}}$ vs. $v_{in}$, with noise margins and their voltages
labeled.\\

\begin{center}
    \includegraphics[scale=0.4]{scope4.png}
\end{center}

The noise margins defined on the datasheet for the CD4007UB transistor chip show that the worst-case margins 
are $V_{IL}$ = 4.5V, and $V_{IH}$ = 0.5V.\\

\subsection*{Conclusions}

\indent\indent In this section, a simple inverter circuit was demonstrated with a CD4007UB integrated circuit containing
three complimentary pairs of NMOS and PMOS transistors. The transfer characteristic was found, and various intrinsic properties 
of this characteristic were determined and shown to be within limits and reasonable values. \\

\newpage

\section*{Task 3}

\subsection*{Objective}
\indent\indent The objective of this task is to create a ring oscillator by daisy-chaining 3 inverters in a loop, 
and to find its oscillation frequency and propagation delay of the inverter circuit.\\

\subsection*{Procedure}

\indent\indent The circuit below was built using a CD4007UB chip by wiring the three inverters together in a loop.\\

\begin{center}
    \includegraphics[scale=0.35]{ringoscillator.png}
\end{center}

The circuit is then measured with an oscilloscope at $v_{in}$ and $v_{out}$ to measure oscillation frequency 
and propagation delay.\\

\subsection*{Results / Calculations}

The circuit was built, and oscillation began on its own, without the aid of a kick start with a resistor and $V_{DD}$.

\newpage

The following is the oscilloscope display showing $v_{in}$ and $v_{out}$. 

\begin{center}
    \includegraphics[scale=0.5]{scope5.png}
\end{center}

The display shows the oscillation frequency $f_{osc}$ to be 4.39 MHz. The ideal oscillation frequency is:

\begin{equation}
    f_{osc} \approx \frac{1}{3} * (t_{PLH} + t_{PHL})
\end{equation}

Where $t_{PLH}$ and $t_{PHL}$ are the propagation delay from low to high states, and high to low 
states respectively, measured by comparing $v_{in}$ to $v_{out}$ and measuring the delay between the two.

Below is an oscilloscope screen capture showing the propagation delay with the cursors.

\begin{center}
    \includegraphics[scale=0.5]{scope6.png}
\end{center}


The measured propagation delay is 38 ns, which when used to calculate ideal oscillation frequency is:

\begin{equation}
    f_{osc} = \frac{1}{3} * (t_{PLH} + t_{PHL}) = \frac{1}{3} * (38 ns + 38 ns) = \frac{1}{3} * (0.000000076) = 4.38 MHz
\end{equation}


With an error of:

\begin{equation}
    \% err = \frac{4.39 MHz - 4.38 MHz}{4.38 MHz} = 0.22\% error
\end{equation}

While this error value can be calculated, the original equation for $f_{osc}$ was an approximation, and thus cannot
be treated as the true ideal value, however, our value was very close to the approximation.



\subsection*{Conclusions}

\indent\indent In this task, a ring oscillator was built and demonstrated using a CD4007UB triple complimentary pair circuit arranged
as three inverters in a loop. The oscillation frequency was measured and verified through the use of propagation delay.


%\newpage

%\printbibliography[title={\Large References}] %Prints out the bibliography sources that you have used in the document.

\end{document}